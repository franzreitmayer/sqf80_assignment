\documentclass[a4paper,12pt]{article}

% Packages
% Use Helvetica (Arial alternative)
% \usepackage{mathptmx} % Times-like font
% \renewcommand{\familydefault}{\sfdefault}
\usepackage{setspace}
\usepackage[utf8]{inputenc} % Encoding
\usepackage[T1]{fontenc}    % For output encoding
\usepackage[ngerman]{babel}
\usepackage{graphicx}       % For images
\usepackage{amsmath}        % For math equations
\usepackage{hyperref}       % For hyperlinks
\usepackage[a4paper, top=3cm, bottom=3cm, left=2.5cm, right=2cm]{geometry}
\usepackage{fancyhdr} %%Fancy Kopf- und Fußzeilen
\usepackage[backend=biber,style=authoryear]{biblatex}
\addbibresource{literatur.bib}

\onehalfspacing






% Document metadata
\title{Systemtheorie: Ursprung, typische Argumentationspfade und kritische Einwände}
\author{Name: Franz Reitmayer \\ Immatrikulationsnummer: 444139}
\date{\today}

\begin{document}

% Title Page
\maketitle
\vfill
\begin{center}
\textbf{Studiengang: MBA Digital Management und Leadership} \\
AKAD University \\
Semester: Winter 24/25 \\
Betreuer: Prof. Dr. Tobias Specker
\end{center}
\vfill
\newpage
\tableofcontents
\newpage


% Main Content
\section{Kontroversen in der Wissenschaftstheorie}
\subsection{Problemhintergrund}
Der Mensch möchte seine Umwelt verstehen und nach seinen Vorstellungen ändern können. Dazu benötigt er Wissen um die Welt erklären und gestalten zu können. Dazu steht ihm ''Alltagswissen'' zur Verfügung. Unter anderem führen aber z.B. Wunschvorstellungen und kognitive Verzerrungen zu Irrtümern, die letzten Endes dazu führen, dass dieses Wissen dem Zweck nicht mehr gerecht wird. Daher hat sich die Wissenschaftstheorie als eigene Wissenschaft entwickelt, die im Gegensatz zum Alltagswissen wissenschaftslich abgesichertes Wissen hervorzubringen, dass definierten Qualitätsstandards genügt um das Verstehen und Gestalten der Welt  ??? Formulierung ???
\subsection{Zielsetzung und Bearbeitungsstruktur}



\section{Terminologische Klärungen und konzeptionelle Fundamente}
\subsection{Wissenschaftstheorie}
\subsection{Pluralismus wissenschaftstheoretischer Konzepte}
\subsection{Theorien, deren Bewertung und Systemtheorien}
\subsection{Ontologie}
\subsection{Epistemologie}


\section{Argumentationspfade ??}
\subsection{Unterschiedliche Systemtheorien}
\subsubsection{Allgemeine Systemtheorie nach Ludwig von Bertalanffy}
\subsubsection{Soziologische Systemtheorie nach Talcott Parsons}
\subsubsection{Soziologische Systemtheorie nach Niklas Luhmann}
\subsubsection{Systemtheorie des Unternehmens nach Hans Ulrich}
\subsubsection{Soziotechnische Systemtheorie nach Günther Ropohl}

\subsection{Offene und geschlossene Systeme}
\subsection{Interdependenz}
\subsection{Selbstregulation}
\subsection{Autopoiesis}
\subsection{Differenzierung von System und Umwelt}
\subsection{Komplexitätsreduktion}
\subsection{Kommunikation als Hauptelement}
\subsection{Selbstreferenziell}
\subsection{Handlung}
\subsection{Funktionale Differenzierung}

\section{Kritische Einwände}
\subsection{Abstrakte Begriffswelt}
\subsection{Supertheorie vs. fehlende Einheitlichkeit}
\subsection{Allerweltszuschreibung}
\subsection{Fehlen Normativer Aussagen}
\subsection{Inkonsistenzen}
\subsection{Selbstdienlichkeit durch Selbstreferenzialität?}
\newpage

\section{Fazit und Zusammenfassung}
\subsection{Fazit und Kritische Würdigung}
\subsection{Ausblick}


% References
\newpage

\nocite{*}
\printbibliography
% \bibliographystyle{plain}
%\bibliography{literatur}
% End of Document
\end{document}