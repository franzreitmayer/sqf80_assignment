\documentclass[a4paper,12pt]{article}

% Packages
% Use Helvetica (Arial alternative)
% \usepackage{mathptmx} % Times-like font
% \renewcommand{\familydefault}{\sfdefault}
\usepackage{setspace}
\usepackage[utf8]{inputenc} % Encoding
\usepackage[T1]{fontenc}    % For output encoding
\usepackage[ngerman]{babel}
\usepackage{graphicx}       % For images
\usepackage{amsmath}        % For math equations
\usepackage{hyperref}       % For hyperlinks
\usepackage[a4paper, top=3cm, bottom=3cm, left=2.5cm, right=2cm]{geometry}
\usepackage{fancyhdr} %%Fancy Kopf- und Fußzeilen
\usepackage[backend=biber,style=authoryear]{biblatex}
\addbibresource{literatur.bib}



\onehalfspacing






% Document metadata

\title{Systemtheorie: Ursprung, typische Argumentationspfade und kritische Einwände}

\date{\today}

\begin{document}
\author{
    Franz Reitmayer \\
    Am Haning 36 \\
    84424 Isen \\
    E-Mail: franz.reitmayer@reitmayer.eu \\ 
    Immatrikulationsnummer: 444139
}

% Title Page
\maketitle
\vfill
\begin{center}
\textbf{Studiengang: MBA Digital Management und Leadership} \\
AKAD University \\
Semester: Winter 24/25 \\
Betreuer: Prof. Dr. Tobias Specker \\

\end{center}
\vfill
\newpage
\tableofcontents
\newpage



% Main Content
\section{Kontroversen in der Wissenschaftstheorie}
\subsection{Problemhintergrund}
Der Mensch möchte seine Umwelt verstehen und nach seinen Vorstellungen gestalten.\footcite[S. 4]{Helfrich2024} Er benötigt Wissen um die Welt erklären und verändern zu können.\footcite[Seite 1]{Helfrich2024} Dazu steht ihm ''Alltagswissen'' zur Verfügung. Unter anderem führen aber z.B. Wunschvorstellungen und kognitive Verzerrungen zu Irrtümern\footcite[S. 8]{Helfrich2024}, die letzten Endes dazu führen, dass dieses Wissen dem Zweck nicht mehr gerecht wird.\footcite[Seite 9]{Helfrich2024} Daher hat sich die Wissenschaftstheorie als eigene Wissenschaft entwickelt, die im Gegensatz zum Alltagswissen wissenschaftslich abgesichertes Wissen hervorzubringen, dass definierten Qualitätsstandards genügt um das Verstehen und Gestalten der Welt zu systematisierten und zu institutonalisieren.\footcite[Seite 4]{Kornmeier2007}\footcite[S. 3]{Helfrich2024} Im Rahmen dieser Professionalisierung des Wissenserwerbs haben sich unterschiedliche Positionen und Theorien entwickelt.\footcite[S. 19 ff]{Kornmesser2020}\footcite[S. 93]{Helfrich2024}. Eine dieser Theorien ist die Systemtheorie\footcite[S. 107]{Helfrich2024}.

\subsection{Zielsetzung und Bearbeitungsstruktur}
Eine Möglichkeit das oben genannte Fundament in wissenschaftlichen Arbeiten umzusetzen bietet die Systemtheorie an. Die Zielsetzung dieser Arbeit ist diese Systemtheorie zu Beschreiben, deren Ursprung und Entwicklung zu skizzieren um abschließend kritische Einwände zu besprechen. Dazu werden im ersten Teil die begrifflichen Grundlagen geschaffen. Die Skizze der Entwicklung wird zeigen, dass es viele unterschiedliche Systemtheorien gibt. Um den Rahmen dieser Arbeit nicht zu sprengen wird dann fortan der Schwerpunkt auf die Systemtheorie nach Hans Ulrich gelegt, da diese Systemtheorie heute weite Anwendung in der BWL und insbesondere der Managementlehre und dem St. Gallener Modell findet.


\section{Terminologische Klärungen und konzeptionelle Fundamente}
\subsection{Wissenschaftstheorie}
Was ist Wissenschaftstheorie, unterschiedliche Ziele von Wissenschaft als Teil der Wissenschaftstheorie, Methoden als Hauptfundament, Dreiklang Ontologie, Epistemologie und Methoden
\subsection{Pluralismus wissenschaftstheoretischer Konzepte}
Enmtstehung und Begründung des Pluralismus der Konzept, auch mit Hinweis auf die Historie und unterschiedliche Ziel
\subsection{Ontologie}
Was ist Ontologie
\subsection{Epistemologie}
Was ist Epistimologie

\subsection{Theorien und deren Bewertung}
Was sind Theorien, unterschiedliche Arten von Theorien, wie kann man Theorien Bewerten

\subsection{Systemtheorien, Ursprung und Arten}
Historie der Systemtheorie, Bertalanffy, Talcot Parson, Luhmann und Hans Ulrich

\section{Systemtheorie nach Ulrich}
\subsection{Unterschiedliche Systemtheorien}
\subsubsection{Allgemeine Systemtheorie nach Ludwig von Bertalanffy}
\subsubsection{Soziologische Systemtheorie nach Talcott Parsons}
\subsubsection{Soziologische Systemtheorie nach Niklas Luhmann}
\subsubsection{Systemtheorie des Unternehmens nach Hans Ulrich}
\subsubsection{Soziotechnische Systemtheorie nach Günther Ropohl}

\subsection{Offene und geschlossene Systeme}
\subsection{Interdependenz}
\subsection{Selbstregulation}
\subsection{Autopoiesis}
\subsection{Differenzierung von System und Umwelt}
\subsection{Komplexitätsreduktion}
\subsection{Kommunikation als Hauptelement}
\subsection{Selbstreferenziell}
\subsection{Handlung}
\subsection{Funktionale Differenzierung}

\section{Kritische Einwände}
\subsection{Abstrakte Begriffswelt}
\subsection{Supertheorie vs. fehlende Einheitlichkeit}
\subsection{Allerweltszuschreibung}
\subsection{Fehlen Normativer Aussagen}
\subsection{Inkonsistenzen}
\subsection{Selbstdienlichkeit durch Selbstreferenzialität?}
\newpage

\section{Fazit und Zusammenfassung}
\subsection{Fazit und Kritische Würdigung}
\subsection{Ausblick}


% References
\newpage




% \nocite{*}
\printbibliography
% \bibliographystyle{plain}
% \bibliography{literatur}
% End of Document
\end{document}