\documentclass[a4paper,12pt]{article}

% Packages
% Use Helvetica (Arial alternative)
% \usepackage{mathptmx} % Times-like font
% \renewcommand{\familydefault}{\sfdefault}
\usepackage{setspace}
\usepackage[utf8]{inputenc} % Encoding
\usepackage[T1]{fontenc}    % For output encoding
\usepackage[ngerman]{babel}
\usepackage{graphicx}       % For images
\usepackage{amsmath}        % For math equations
\usepackage{hyperref}       % For hyperlinks
\usepackage[a4paper, top=3cm, bottom=3cm, left=2.5cm, right=2cm]{geometry}
\usepackage{fancyhdr} %%Fancy Kopf- und Fußzeilen
\usepackage[backend=biber,style=authoryear-ibid]{biblatex}
\addbibresource{literatur.bib}



\onehalfspacing







% Document metadata

\title{Systemtheorie: Ursprung, typische Argumentationspfade und kritische Einwände}

\date{\today}

\begin{document}
\author{
    Franz Reitmayer \\
    Am Haning 36 \\
    84424 Isen \\
    E-Mail: franz.reitmayer@reitmayer.eu \\ 
    Immatrikulationsnummer: 444139
}

% Title Page
\maketitle
\vfill
\begin{center}
\textbf{Studiengang: MBA Digital Management und Leadership} \\
AKAD University \\
Semester: Winter 24/25 \\
Betreuer: Prof. Dr. Tobias Specker \\

\end{center}
\vfill
\newpage
\tableofcontents
\newpage



% Main Content
\section{Kontroversen in der Wissenschaftstheorie}
\subsection{Problemhintergrund}
Der Mensch möchte seine Umwelt verstehen und nach seinen Vorstellungen gestalten. Er benötigt Wissen, um die Welt erklären und verändern zu können.\footcite[S. 1]{Helfrich2024} Dazu steht ihm ''Alltagswissen'' zur Verfügung. Unter anderem führen aber z.B. Wunschvorstellungen und kognitive Verzerrungen zu Irrtümern, die letzten Endes bedingen, dass dieses Wissen dem Zweck nicht mehr hinreichend gerecht wird.\footcite[S. 9]{Helfrich2024} Daher hat sich die Wissenschaftstheorie als eigene Wissenschaft entwickelt, die im Gegensatz zum Alltagswissen wissenschaftlich abgesichertes Wissen hervorzubringen, dass definierten Qualitätsstandards genügt, um das Verstehen und Gestalten der Welt zu systematisieren und zu institutionalisieren.\footcite[Seite 4]{Kornmeier2007}\footcite[S. 3]{Helfrich2024} Im Rahmen dieser Professionalisierung des Wissenserwerbs haben sich unterschiedliche Positionen und Theorien entwickelt.\footcite[S. 19 ff]{Kornmesser2020}\footcite[S. 93]{Helfrich2024} Eine dieser Theorien ist die Systemtheorie\footcite[S. 107]{Helfrich2024}.

\subsection{Zielsetzung und Bearbeitungsstruktur}
Eine Möglichkeit das oben genannte Fundament in wissenschaftlichen Arbeiten umzusetzen bietet die Systemtheorie an. Die Zielsetzung dieser Arbeit ist diese Systemtheorie zu Beschreiben, deren Ursprung und Entwicklung zu skizzieren um abschließend kritische Einwände zu besprechen. Dazu werden im ersten Teil die begrifflichen Grundlagen geschaffen. Die Skizze der Entwicklung wird zeigen, dass es viele unterschiedliche Systemtheorien gibt. Hier wird explikativ nach einer Definition gesucht, was man sich unter einer Systemtheorie vorstellen kann und im Anschluss deskritiv nach ausgewählten Eigenschaften der Systemtheorie gesucht. Danach wird aus diesem Kontext eine Einordnung in die Kategorien Ontologie, Epistemologie und Methodik der Wissenschaftstheorie versucht. Um den Rahmen dieser Arbeit nicht zu sprengen, liegt der Schwerpunkt im Allgemeinen auf der Systemtheorie in der Betriebwirtschaft und im Speziellen auf der Systemtheorie nach Hans Ulrich. Diese Systemtheorie findet heute Anwendung in der BWL\footcite[S. 26]{Woehe2008} und insbesondere in der Managementlehre, sowie dem St. Gallener Management-Modell.\footcite[S. 10]{RueeggStuerm2020}.



\section{Terminologische Klärungen und konzeptionelle\\ Fundamente}
\subsection{Wissenschaftstheorie}
% Was ist Wissenschaftstheorie, unterschiedliche Ziele von Wissenschaft als Teil der Wissenschaftstheorie, Methoden als Hauptfundament, Dreiklang Ontologie, Epistemologie und Methoden
Um zu erötern was Wissenschaftstheorie ist, kann man die Frage nach der Wissenschaft voranstellen. Grundsätzlich beginnen die Fragenstellungen zu Wissen, Wissenschaft und Wissenschafttheorie in der Philosophie der griechischen Antike.\footcite[S. 3]{Kornmesser2020}\footcite[S. 686]{Reicher2023} Im Laufe der Zeit trennten sich die philosophischen Fragestellungen zur Wissenschaft in unterschiedliche Teildisziplinen auf, wovon eine die Wissenschaftstheorie ist.\footcite[S. 4]{Kornmesser2020} Heute kann der Begriff Wissenschaft institutional, ergebnisorientiert oder tätigkeits- bzw. aufgabenorientiert aufgefasst werden.\footcite[S. 4]{Kornmeier2007} Aus dieser Entwicklung heraus beschäftigt sich die Wissenschaftstheorie heute vor allem mit den grundlegenden Fragen, was Wissenschaft ist, wie man strukturiert zu wissenschaftlicher Erkenntnis kommt, welche Methoden in den Wissenschaften wie eingesetzt werden können und wie wissenschaftliche Aussagen begründet werden können.\footcite[S. 5]{Kornmesser2020} Ziel dabei ist es im Unterschied zur Alltagserkenntnis bestimmte Normen einzuhalten, um wisschenschaftlicher Überprüfung Stand zu halten und damit gewisse Qualitätsstandards sicherzustellen.\footcite[S. 9]{Helfrich2024} Wenn es die Anwendung in einem Fachgebiet order einer Gruppe von Wissenschaften erfordert trennt sich die Wissenschaftstheorie weiter in eine allgemeine und für den Fachbereich spezielle Wisschenschaftstheorie.\footcite[S. 5]{Kornmesser2020} Dies ist z.B. in der Betriebswirtschaftslehre bzw. den Wirschaftswissenschaften der Fall.\footcite[S. 25]{Helfrich2024}



\subsection{Pluralismus wissenschaftstheoretischer Konzepte}
% Entstehung und Begründung des Pluralismus der Konzept, auch mit Hinweis auf die Historie und unterschiedliche Ziel
% Konrmeier S 104 - 105
% Kornmesser 161
% Auswahlprinzip bei Wöhe
% helfrich S 94 --> Ontologie und Epistemologie
Aus der oben gezeigten gezeigten Spezialisierung der Wissenschaftstheorie wird bereits der Pluralismus deutlich, der letzten Endes auch auf die angewendeten Konzepte und Methoden durchschlagen muss. Zieht man die unterschiedlichen Aufgaben der Betriebswirtschaftslehre, Beschreiben, Erklären, Vorhersagen und das Gestalten von Handlungsmaßnahmen hinzu\footcite[S. 25]{Helfrich2024}, dann wird deutlich, dass es nicht nur ein Konzept geben kann mit dem all diese Anforderungen aus- und hinreichend erfüllt werden können. Diese Pluralität zeigt sich in unterschiedlichen wissenschaftstheoretischen Positionen\footcite[S. 93]{Helfrich2024}, die sich "'im Wesentlichen auf zwei Dimensionen ansiedeln"'\footcite[S. 94]{Helfrich2024} lassen, der Onotologischen und der Epistimologischen Dimension.\footcite[S. 94]{Helfrich2024}\footcite[S. 29]{Kornmeier2007} Diese Zuordnung ist aber vereinfacht, da "'die beiden Dimensionen (Rationalismus vs. Empirismus und Realismus vs. Konstruktivismus) streng genommen nicht unabhängig voneinander sind"'\footcite[S. 29]{Kornmeier2007}. Diese beiden Dimensionen werden im folgendem nocg genauer erklärt.
Ein weiterer Hinweis auf die Bedeutung der Untersuchungsperspektive und die damit einhergehende Notwendigkeit unerschiedlicher Positionen ergibt aus dem Auswahlprinzip bei Woehe\footcite[S. 38]{Woehe2008}. Selbst innerhalb der Allgemeinen Betriebswirtschaftslehre müssen erste Annahmen zu Erfahrungs- und Erkenntnisobjekt getroffen werden um die weitere Bearbeitung des Themas zu lenken. Eine dieser möglichen wissenschaftstheoretischen Positionen ist die Systemtheorie.

\subsection{Ontologie}
% Was ist Ontologie
In der Ontologie wird die Rezeption und Erfahrbarkeit des Forschungsgegenstandes beschrieben. "'Die ontologischen Dimensionen mit den beiden Polen 'realistisch' und 'idealistisch' bezieht sich darauf, wie man sich die Beschaffenheit  der Wirklichkeit vorstellt"'\footcite[S. 94]{Helfrich2024}. Der Pol realistisch trifft z.B. auf die im Realismus vertretene Ansicht zu, dass es eine vom Beobachter unabhängige Realität gibt.\footcite[S. 95]{Helfrich2024} Dagegen fordert die Annahme des klassischen Rationalismus, "'dass wissenschaftliche Erkenntnis nicht durch sinnliche Erfahrung gewonnen werden kann, sondern auf verstandesgemäßen bzw. logischen Überlegungen beruht"'\footcite[S. 100]{Helfrich2024} auf der ontologischen Dimension eine Abstraktion vom Erfahrungsgegenstand und ordnet diese damit am Pol idealistisch ein.
\subsection{Epistemologie}
% Was ist Epistimologie
Die Epistemologie ist eigentlich eine philosophische Teildisziplin (allgemeine Erkenntnistheorie).\footcite[S. 1]{Helfrich2024}. Hier ist die Bedeutung vor allem wie und auf welche Weise ein Erkenntnisgewinn überhaupt möglich ist. Im klassischen Empirismus ist beispielsweise Erkenntnis nur durch sinnliche Erfahrung möglich, die physikalische Beschaffenheit der Dinge spielt dabei keine Rolle.\footcite[S. 98]{Helfrich2024}. Damit ist der Empirismus auf der epistemologischen Achse dem Pol empirisch zuzuordnen. Die oben bereits erwähnte Position des klassischen Rationalismus dem Pol rational.
\subsection{Theorien}
% Was sind Theorien, unterschiedliche Arten von Theorien, wie kann man Theorien Bewerten
Es "'können zwei große Positionen der Theorienkonstruktion und -analyse unterschieden werden'"\footcite[S. 100]{Kornmesser2020} Es existiert zum einen die Aussagenkonzeption, die eine Theorie als Menge von Sätzen auffasst.\footcite[S. 100]{Kornmesser2020} und zum anderen der Strukturalismus der eine Theorie modelltheoretisch als Menge von Modellen beschreibt.\footcite[S. 100]{Kornmesser2020} In der Betriebswirtschaftslehre scheint zunächst die Position der Aussagenkonzeption vorherrschend.\footcite[S. 61]{Helfrich2024} Allerdings ergibt sich aus der Verbindung von Modell und Theorie eine Mischform, die zuerst Aussagen zu Hypothesen zusammenführt, diese Hypothesen logisch zu Modellen verbindet und die unterschiedlichen Modelle abschließend zu einer Theorie integriert.\footcite[S. 84]{Kornmeier2007}. Diese Integration von Hypothesen findet sich jedoch auch in der auf der Aussagenkonzeption aufbauenden Position wieder\footcite[S. 63]{Helfrich2024}, weshalb hier im weiteren von der vorgenannten Mischform ausgeganden wird.
\section{Systemtheorien}
\subsection{Unterschiedliche Systemtheorien und historische Entwicklung}
%\subsubsection{Allgemeine Systemtheorie nach Ludwig von Bertalanffy}
%\subsubsection{Soziologische Systemtheorie nach Niklas Luhmann}
%\subsubsection{Systemtheorie des Unternehmens nach Hans Ulrich}
%\subsubsection{Soziotechnische Systemtheorie nach Günther Ropohl}

\subsection{Teil und Ganzheit}
Ulrich 107
\subsection{Systemstruktur}
ulrich 109
\subsection{Offene und geschlossene Systeme}
Ulrich 112
\subsection{Systemdynamik}
\subsection{Autopoiesis und Freiheitsgrade}
\subsection{Zweck und Zielorientierung}
\subsection{Determiniertheit und Probablistik}
\subsection{Wissenschaftstheoretische Einordnung}
siehe Wissenschaftsprogramm%20und%20Ausbildungsziele%20der%20Betriebswirtschaftslehre.pdf Seite 45

\section{Kritische Einwände}
\subsection{Abstrakte Begriffswelt}
\subsection{Supertheorie vs. fehlende Einheitlichkeit}
\subsection{Allerweltszuschreibung}
\subsection{Fehlen Normativer Aussagen}
\subsection{Inkonsistenzen}
\subsection{Selbstdienlichkeit durch Selbstreferenzialität?}
\newpage

\section{Fazit und Zusammenfassung}
\subsection{Fazit und Kritische Würdigung}
\subsection{Ausblick}


% References
\newpage




% \nocite{*}
\printbibliography
% \bibliographystyle{plain}
% \bibliography{literatur}
% End of Document
\end{document}