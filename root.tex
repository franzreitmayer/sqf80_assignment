\documentclass[a4paper,12pt]{article}

% Packages
% Use Helvetica (Arial alternative)
% \usepackage{mathptmx} % Times-like font
% \renewcommand{\familydefault}{\sfdefault}
\usepackage{setspace}
\usepackage[utf8]{inputenc} % Encoding
\usepackage[T1]{fontenc}    % For output encoding
\usepackage[ngerman]{babel}
\usepackage{graphicx}       % For images
\usepackage{amsmath}        % For math equations
\usepackage{hyperref}       % For hyperlinks
\usepackage[a4paper, top=3cm, bottom=3cm, left=2.5cm, right=2cm]{geometry}
\usepackage{fancyhdr} %%Fancy Kopf- und Fußzeilen
\usepackage[backend=biber,style=authoryear-ibid]{biblatex}
\usepackage[hang]{footmisc}
\usepackage{csquotes}
\addbibresource{literatur.bib}



\onehalfspacing


% Document metadata

\title{Systemtheorie: Ursprung, typische Argumentationspfade und kritische Einwände}

\date{\today}

\begin{document}
\author{
    Franz Reitmayer \\
    Am Haning 36 \\
    84424 Isen \\
    E-Mail: franz.reitmayer@reitmayer.eu \\ 
    Immatrikulationsnummer: 444139
}

% Title Page
\maketitle
\vfill
\begin{center}
\textbf{Studiengang: MBA Digital Management und Leadership} \\
AKAD University \\
Semester: Winter 24/25 \\
Betreuer: Prof. Dr. Tobias Specker \\

\end{center}
\vfill
\newpage
\tableofcontents
\newpage



% Main Content
\section{Kontroversen in der Wissenschaftstheorie}
\subsection{Problemhintergrund}
Der Mensch möchte seine Umwelt verstehen und nach seinen Vorstellungen gestalten. Er benötigt Wissen, um die Welt erklären und verändern zu können.\footcite[S. 1]{Helfrich2024} Dazu steht ihm ''Alltagswissen'' zur Verfügung. Unter anderem führen aber z.B. Wunschvorstellungen und kognitive Verzerrungen zu Irrtümern, die letzten Endes bedingen, dass dieses Wissen dem Zweck nicht mehr hinreichend gerecht wird.\footcite[S. 9]{Helfrich2024} Daher hat sich die Wissenschaftstheorie als eigene Wissenschaft entwickelt, die im Gegensatz zum Alltagswissen wissenschaftlich abgesichertes Wissen hervorzubringt, das definierten Qualitätsstandards genügt, um das Verstehen und Gestalten der Welt zu systematisieren und zu institutionalisieren.\footcite[Seite 4]{Kornmeier2007}\footcite[S. 3]{Helfrich2024} Im Rahmen dieser Professionalisierung des Wissenserwerbs haben sich unterschiedliche Positionen und Theorien entwickelt.\footcite[S. 19 ff]{Kornmesser2020}\footcite[S. 93]{Helfrich2024} Eine dieser Theorien ist die Systemtheorie\footcite[S. 107]{Helfrich2024}.

\subsection{Zielsetzung und Bearbeitungsstruktur}
Eine Möglichkeit das oben genannte Fundament in wissenschaftlichen Arbeiten umzusetzen bietet die Systemtheorie an. Die Zielsetzung dieser Arbeit ist diese Systemtheorie zu Beschreiben, deren Ursprung und Entwicklung zu skizzieren, um abschließend kritische Einwände zu besprechen. Dazu werden im ersten Teil die begrifflichen Grundlagen geschaffen. Die Skizze der Entwicklung wird zeigen, dass es viele unterschiedliche Systemtheorien gibt. Hier wird explikativ nach einer Definition gesucht, was man sich unter einer Systemtheorie vorstellen kann und im Anschluss deskriptiv nach ausgewählten Eigenschaften der Systemtheorie gesucht. Danach wird aus diesem Kontext eine Einordnung in die Kategorien Ontologie und Epistemologie versucht. Um den Rahmen dieser Arbeit nicht zu sprengen, liegt der Schwerpunkt im Allgemeinen auf der Systemtheorie in der Betriebwirtschaft und im Speziellen auf der Systemtheorie nach Hans Ulrich. Diese Systemtheorie findet heute Anwendung in der BWL\footcite[S. 26]{Woehe2008} und insbesondere in der Managementlehre, sowie dem St. Gallener Management-Modell.\footcite[S. 10]{RueeggStuerm2020}.



\section{Terminologische Klärungen und konzeptionelle\\ Fundamente}
\subsection{Wissenschaftstheorie}
% Was ist Wissenschaftstheorie, unterschiedliche Ziele von Wissenschaft als Teil der Wissenschaftstheorie, Methoden als Hauptfundament, Dreiklang Ontologie, Epistemologie und Methoden
Um zu erötern was Wissenschaftstheorie ist, kann man die Frage nach der Wissenschaft voranstellen. Grundsätzlich beginnen die Fragestellungen zu Wissen, Wissenschaft und Wissenschaftstheorie in der Philosophie der griechischen Antike.\footcite[S. 3]{Kornmesser2020}\footcite[S. 686]{Reicher2023} Im Laufe der Zeit trennten sich die philosophischen Fragestellungen zur Wissenschaft in unterschiedliche Teildisziplinen auf, wovon eine die Wissenschaftstheorie ist.\footcite[S. 4]{Kornmesser2020} Heute kann der Begriff Wissenschaft institutional, ergebnisorientiert oder tätigkeits- bzw. aufgabenorientiert aufgefasst werden.\footcite[S. 4]{Kornmeier2007} Aus dieser Entwicklung heraus beschäftigt sich die Wissenschaftstheorie heute vor allem mit den grundlegenden Fragen, was Wissenschaft ist, wie man strukturiert zu wissenschaftlicher Erkenntnis kommt, welche Methoden in den Wissenschaften wie eingesetzt werden können und wie wissenschaftliche Aussagen begründet werden können.\footcite[S. 5]{Kornmesser2020} Ziel dabei ist es im Unterschied zur Alltagserkenntnis bestimmte Normen einzuhalten, um wisschenschaftlicher Überprüfung Stand zu halten und damit gewisse Qualitätsstandards sicherzustellen.\footcite[S. 9]{Helfrich2024} Wenn es die Anwendung in einem Fachgebiet oder einer Gruppe von Wissenschaften erfordert, trennt sich die Wissenschaftstheorie weiter in eine allgemeine und für den Fachbereich spezielle Wisschenschaftstheorie.\footcite[S. 5]{Kornmesser2020} Dies ist z.B. in der Betriebswirtschaftslehre bzw. den Wirtschaftswissenschaften der Fall.\footcite[S. 25]{Helfrich2024}



\subsection{Pluralismus wissenschaftstheoretischer Konzepte}\label{plural}
% Entstehung und Begründung des Pluralismus der Konzept, auch mit Hinweis auf die Historie und unterschiedliche Ziel
% Konrmeier S 104 - 105
% Kornmesser 161
% Auswahlprinzip bei Wöhe
% helfrich S 94 --> Ontologie und Epistemologie
Aus der oben gezeigten Spezialisierung der Wissenschaftstheorie wird bereits der Pluralismus deutlich, der letzten Endes auch auf die angewendeten Konzepte und Methoden durchschlagen muss. Zieht man die unterschiedlichen Aufgaben der Betriebswirtschaftslehre: Beschreiben, Erklären, Vorhersagen und das Gestalten von Handlungsmaßnahmen hinzu\footcite[S. 25]{Helfrich2024}, dann wird deutlich, dass es nicht nur ein Konzept geben kann mit dem all diese Anforderungen aus- und hinreichend erfüllt werden können. Diese Pluralität zeigt sich in unterschiedlichen wissenschaftstheoretischen Positionen\footcite[S. 93]{Helfrich2024}, die sich "'im Wesentlichen auf zwei Dimensionen ansiedeln"'\footcite[S. 94]{Helfrich2024} lassen: der ontologischen und der epistemologischen Dimension.\footcite[S. 94]{Helfrich2024}\footcite[S. 29]{Kornmeier2007} Diese Zuordnung ist aber vereinfacht, da "'die beiden Dimensionen (Rationalismus vs. Empirismus und Realismus vs. Konstruktivismus) streng genommen nicht unabhängig voneinander sind"'\footcite[S. 29]{Kornmeier2007}. Diese beiden Dimensionen werden im Folgenden noch genauer erklärt.\footnote{Die beiden Pole Rationalismus und Empirismus auf der Epistemischen Achse werden im folgenden mit rational und empirisch beschrieben, auf der Ontologischen Achse Realismus und Konstruktivismus mit realistisch und idealistisch}
Ein weiterer Hinweis auf die Bedeutung der Untersuchungsperspektive und die damit einhergehende Notwendigkeit unterschiedlicher Positionen ergibt sich aus dem Auswahlprinzip bei Woehe\footcite[S. 38]{Woehe2008}. Dies ist wie folgt definiert:
\begin{quote}
"'Als \textbf{Auswahlprinzip} bezeichnet man die zur Erforschung des Erfahrungsobjektes (Betrieb) eingenommene Untersuchungsperspektive."'\footcite[S. 38]{Woehe2008}    
\end{quote} 
Selbst innerhalb der Allgemeinen Betriebswirtschaftslehre müssen also erste Annahmen zu Erfahrungs- und Erkenntnisobjekt getroffen werden, um die weitere Bearbeitung des Themas zu lenken. Eine dieser möglichen wissenschaftstheoretischen Positionen ist die Systemtheorie.

\subsection{Ontologie}
% Was ist Ontologie
In der Ontologie wird die Rezeption und Erfahrbarkeit des Forschungsgegenstandes beschrieben. "'Die ontologischen Dimensionen mit den beiden Polen 'realistisch' und 'idealistisch' beziehen sich darauf, wie man sich die Beschaffenheit  der Wirklichkeit vorstellt"'\footcite[S. 94]{Helfrich2024}. Der Pol realistisch trifft z.B. auf die im Realismus vertretene Ansicht zu, dass es eine vom Beobachter unabhängige Realität gibt.\footcite[S. 95]{Helfrich2024} Dagegen fordert die Annahme des klassischen Rationalismus, "'dass wissenschaftliche Erkenntnis nicht durch sinnliche Erfahrung gewonnen werden kann, sondern auf verstandesgemäßen bzw. logischen Überlegungen beruht"'\footcite[S. 100]{Helfrich2024}. Daher fordert der Rationalismus auf der ontologischen Dimension eine Abstraktion vom Erfahrungsgegenstand und ordnet diese damit am Pol idealistisch ein.
\subsection{Epistemologie}
% Was ist Epistimologie
Die Epistemologie ist eigentlich eine philosophische Teildisziplin (allgemeine Erkenntnistheorie)\footcite[S. 1]{Helfrich2024}. Hier ist die Bedeutung vor allem wie und auf welche Weise ein Erkenntnisgewinn überhaupt möglich ist. Im klassischen Empirismus ist beispielsweise Erkenntnis nur durch sinnliche Erfahrung möglich, die physikalische Beschaffenheit der Dinge spielt dabei keine Rolle.\footcite[S. 98]{Helfrich2024} Damit ist der Empirismus auf der epistemologischen Achse dem Pol empirisch zuzuordnen. Die oben bereits erwähnte Position des klassischen Rationalismus dem Pol rational.
\subsection{Theorien}
% Was sind Theorien, unterschiedliche Arten von Theorien, wie kann man Theorien Bewerten
Es "'können zwei große Positionen der Theorienkonstruktion und -analyse unterschieden werden"'\footcite[S. 100]{Kornmesser2020} Es existiert zum einen die Aussagenkonzeption, die eine Theorie als Menge von Sätzen auffasst\footcite[S. 100]{Kornmesser2020} und zum anderen der Strukturalismus der eine Theorie modelltheoretisch als Menge von Modellen beschreibt.\footcite[S. 100]{Kornmesser2020} In der Betriebswirtschaftslehre scheint zunächst die Position der Aussagenkonzeption vorherrschend.\footcite[S. 61]{Helfrich2024} Allerdings ergibt sich aus der Verbindung von Modell und Theorie eine Mischform, die zuerst Aussagen zu Hypothesen zusammenführt, diese Hypothesen logisch zu Modellen verbindet und die unterschiedlichen Modelle abschließend zu einer Theorie integriert.\footcite[S. 84]{Kornmeier2007} Diese Integration von Hypothesen findet sich jedoch auch in der auf der Aussagenkonzeption aufbauenden Position wieder\footcite[S. 63]{Helfrich2024}, weshalb hier im Weiteren von der vorgenannten Mischform ausgegangen wird.
\section{Systemtheorien}
\subsection{Unterschiedliche Systemtheorien und historische Entwicklung}
%\subsubsection{Allgemeine Systemtheorie nach Ludwig von Bertalanffy}
%\subsubsection{Soziologische Systemtheorie nach Niklas Luhmann}
%\subsubsection{Systemtheorie des Unternehmens nach Hans Ulrich}
%\subsubsection{Soziotechnische Systemtheorie nach Günther Ropohl}
Befasst man sich mit der Systemtheorie, so fällt schnell auf, dass schon der Begriff System durch ein breites Spektrum an Definitionen bestimmt wird.\footcite[S. 18]{Patzak1982} Immer wieder haben Philosophen und Wissenschaftler sich um eine "'formale, ganzheitliche Betrachtung eines weiten Objektbereiches bemüht"'\footcite[S. 102]{Ulrich1968}. Diese Tätigkeit wurde Teils aus unterschiedlichen Fachgebieten und teilweise zur gleichen Zeit parallel und ohne Abstimmung geleistet. Damit sind viele unterschiedliche Ansätze und Untersuchungsziele in den Begriff eingeflossen. "'\textit{Systemtheorie} kann demnach auch nicht als eine einheitliche Theorie bezeichnet werden. Die Unterschiede zwischen den systemtheoretischen Ansätzen sind so groß, dass Hans Lenk sogar von einem '[...] Sammelreservoir theoretisch und methodologisch unterschiedlicher, disziplinübergreifender, aber durch Projektbezogenheit verbundener Modellansätze[...]' (Lenk 1978, 245) sprechen konnte."'\footcite[S. 111]{Diesner2015}
Diese Arbeit stützt sich im Weiteren auf die von Hans Ulrich mit der "'Unternehmung als produktives soziales System"'\footcite[S. 3]{Ulrich1968} eingeführten Systemtheorie.
Dieser allgemeine Systemtheoriebegriff umfasst auch die allgemeine Kybernetik\footcite[S. 102]{Ulrich1968} die insbesondere Beiträge zur Steuerung und von Regelkreisen in den Systemen liefert.\footcite[S. 50]{Ulrich2016} Die Grundlagen für eine allgemeine Systemtheorie wurden vor allem von dem Biologen Ludwig von Bertalanffy geschaffen.\footcite[S. 102]{Ulrich1968} Dieser erstellte ausgehend von einem organismischen Ansatz die erste allgemeine Theorie offener Systeme\footcite[S. 70]{Diesner2015} und formalisierte und abstrahierte diese Theorie vor dem Hintergrund des dringenden Bedarfs eines interdisziplinären Ansatzes zur allgemeinen Systemtheorie, die in der Lage ist unterschiedliche disziplinäre Ansätze zu integrieren.\footcite[S. 71]{Diesner2015} Viele darauf folgende Systemtheorien beziehen sich u.a. auf diese allgemeine Systemtheorie. So z.B. die Systemtheorie nach Luhmann\footcite[S. 15]{Luhmann1999} oder auch nach Günther Ropohl\footcite[S. 70]{Ropohl2009}.

\subsection{Teil und Ganzheit}\label{teilganzheit}
% Ulrich 107
Ein wichtiges Begriffsmerkmal der Systemtheorie ist der von "Teil und Ganzheit".\footcite[S. 107]{Ulrich1968} In Folge dieser Betrachtungsweise werden die Begriffe Subsystem und Supersystem eingeführt. Die zu betrachtende Ganzheit, die den eigentlichen Untersuchungsgegenstand beschreibt, wird als System bezeichnet\footcite[S. 107]{Ulrich1968}, "'das möglicherweise Bestandteil eines größeren 'Supersystems'"'\footcite[107]{Ulrich1968} ist. "Teile des Systems können als Subsystem aufgefasst werden."\footcite[S. 107]{Ulrich1968} Teile die man nicht weiter aufteilen kann oder will werden als Element aufgefasst. Ist eine weitere Analyse aber nötig, so kann das Element wiederum zum System interpretiert werden, so lange die Merkmale des Systembegriffs darauf anwendbar sind.\footcite[S. 107]{Ulrich1968} Damit ist die Durchgängigkeit und Erweiterbarkeit des Systembegriffs und der Systemtheorie in dem Merkmal Teil und Ganzheit für das Untersuchungsobjekt gewährleistet. 
\subsection{Interdependenzen}\label{interdep}
Zwischen den oben beschriebenen Super-, Subsystemen und Elementen bestehen unterschiedliche Beziehungen. "'Was unter 'Beziehungen' allgemein zu verstehen ist, scheint nicht restlos geklärt zu sein"'\footcite[109]{Ulrich1968}. Daher wurde hier dafür auch der Begriff Interdependenz gewählt. Dieses Konzept bildet nämlich die Grundlage für die so genannten zirkulären Kausalbeziehungen\footcite[S. 104]{Helfrich2024}, die den größten Unterschied der Systemthoerie zu vielen anderen Ansätzen bildet. Statt einer linearen Kausalbeziehung aus Ursache und Wirkung wird hier die gegenseitige Wechselwirkung der Beziehung betont. In einer Ursache-Wirkungsbeziehung würde man nur den Fluss von Information, Materie oder Energie als Ursache betrachten und am Ziel der Beziehung die Wirkung analysieren. In der Systemtheorie muss aber festgestellt werden, dass die Interaktion wieder auf den Ausgangspunkt zurückwirkt und auch dort Wirkungen entfaltet, die wiederum erneut andere Auswirkungen haben können.\footcite[S. 104]{Helfrich2024} 
\subsection{Systemstruktur}\label{systemstruktur}
% ulrich 109
Die durch die oben beschriebenen Elemente, Systeme, Sub- und Supersysteme und deren Beziehungen bzw. Interdependenzen entstehende Anordnung wird Systemstruktur genannt.\footcite[S. 110]{Ulrich1968} Diese Struktur bildet die Ordnung des Systems. Mit diesen Argumentationen läßt sich nun allgemein definieren was unter einem System verstanden werden kann:
\begin{quote}
    Unter einem System verstehen wir eine geordnete Gesamtheit von Elementen, zwischen denen irgendwelche Beziehungen bestehen oder hergestellt werden können.\footcite[S. 105]{Ulrich1968}
\end{quote}
\subsection{Offene und geschlossene Systeme}\label{offengeschlossen}
% Ulrich 112
Ein System kann nach aussen offen oder geschlossen sein. Wenn das System wie in \ref{teilganzheit}\footnote{S. \pageref{teilganzheit}} beschrieben Teil einer größeren Ganzheit, einer Umwelt ist, dann besteht bereits eine Beziehung nach aussen und das System ist ein so genanntes offenes System.\footcite[S. 112]{Ulrich1968} Vollständige Geschlossenheit und vollständige Offenheit bilden die Skala an Möglichkeiten von Offenheit und Geschlossenheit ab. Die vollständige Geschlossenheit scheint zwar theoretisch möglich, ist aber praktisch nicht relevant. Das andere Extrem: die vollständige Offenheit kann es theoretisch nicht geben bzw. würde diese keinen Sinn ergeben. Sie würde bedeuten, dass ein System mindestens so viele Beziehungen nach aussen wie nach innen hätte und damit vollständig in seiner Umwelt aufgehen und damit seine Identität verlieren würde.\footcite[S.112 - 113]{Ulrich1968} 
\subsection{Systemdynamik}\label{systemdynamik}
Betrachtet man die in \ref{systemstruktur}\footnote{S. \pageref{systemstruktur}} beschriebene Systemstruktur in ihrem Verlauf über die Zeit, stellt man fest, dass diese nicht statisch sein muss. In der zeitlichen Dimension bildet eine bestimmte Abfolge von Interdependenzen ein Muster ab, dass man als Prozessstruktur bezeichnet.\footcite[S. 110]{Ulrich1968} Im Rahmen dieser dynamischen Systemperspektive ist auch noch von Interesse, ob sich diese Abfolge von Aktivitäten gegenüber Supersystemen abspielt oder in den Binnenstrukturen des Systems verbleibt. Im ersten Fall spricht man von äußerer Dynamik und im zweiten Fall von Verhalten oder Aktivität.\footcite[S. 113]{Ulrich1968} Gerade für die Analyse von Systemen sind diese Prozessstrukturen relevant. Im Verhalten für die Prozesseanalyse\footcite[S. 114]{Ulrich1968} und in der äußeren Dynamik auch in der Organisationsgestaltung.

\subsection{Zweck und Zielorientierung}\label{zweckundziel}
In der Handlungsperspektive werden Systeme vom Menschen selbst erschaffen. In diesem Fall soll die Dynamik der Systeme nicht ziellos sein. Innerhalb der Systemtheorie meint Zweck bestimmte Funktionen, die ein System in seiner Umwelt ausübt und Ziel eine vom System selbst angestrebte Verhaltensweise.\footcite[S. 114]{Ulrich1968} "'In der Regel wird ein bestimmter Output erwartet, den das System irgendwelchen Elementen der Umwelt bereitstellen soll. Die Dynamik oder das Verhalten des Systems soll also zielgerichtet sein."'\footcite[S. 114]{Ulrich1968} Natürliche Systeme folgen zwar keiner menschlichen Zielsetzung sehr wohl können Sie aber für den Menschen nützlichen Output bereitstellen oder Funktionen ausüben. In diesem Fall wird der Zweck durch ein entsprechendes Analyseziel definiert.\footcite[S. 114]{Ulrich1968}
\subsection{Autopoiesis und Freiheitsgrade}\label{selbsterhalt}
Wie oben schon beschrieben können Systeme auch im Rahmen eigenen Ermessens selbst Ziele definieren und diese verfolgen.\footcite[S. 114]{Ulrich1968} Diese Freiheitsgrade zuzusprechen kann durchaus Sinn ergeben, vor allem in der Organisationsgestaltung. Wie in \ref{offengeschlossen} Offene und geschlossene Systeme angesprochen, können Systeme auch vom Niedergang betroffen sein. Die beschriebene vollständige Offenheit war dazu nur ein Beispiel. Ist ein System in der Lage den Austausch mit seiner Umwelt in Form von Output und Input und Aufbau und Zerfall der eigenen Elemente durch ständige Veränderung so im Gleichgewicht zu halten, dass seine Existenz gesichert bleibt\footcite[S. 113]{Ulrich1968}, so spricht man vom so genannten Fließgleichgewicht.\footcite[86]{Diesner2015} Im Rahmen von soziologischen Systemen spricht man hier auch von Autopoiesis.\footcite[S. 63]{Willke2006}\footnote{Hier weichen wir Ausnahmsweise von der ausschließlichen Anlehnung an Hans Ulrich ab, da dieser Punkt gerade hohe Aktualität in der Gestaltung agiler Teams hat} Gundlage für diese Resilienz von Systemen ist also die Offenheit, die Veränderung und die Fähigkeit und der Freiheitsgrade zur Gestaltung im Inneren im Sinne des Fließgleichgewichts, die eine evolutionäre Entwicklung von Systemen erlauben.\footcite[144]{Malik2009}
 

\subsection{Wissenschaftstheoretische Einordnung}\label{einordnung}
% siehe Wissenschaftsprogramm%20und%20Ausbildungsziele%20der%20Betriebswirtschaftslehre.pdf Seite 45
Ziel der bisherigen Ausführungen war an dieser Stelle eine ontologische und epistemologische Einordnung der oben beschriebenen Systemtheorie vornehmen zu können. Betrachten wir dazu zunächst die ontologische Ebene. Die Darstellung der Systemtheorie zeigt, dass diese im allgemeinen Fall ihre Untersuchungsobjekte zu Systemen, Sub- und Supersystemen, sowie Elementen abstrahieren muss und danach untersuchungsrelevante Attribute oder Modelle auf diesen Ebenen zuschreibt und die Analyse an Hand dieser Objekte weiterführt. Dafür spielen u.a. \ref{teilganzheit} Teil- und Ganzheit\footnote{S. \pageref{teilganzheit}}, \ref{interdep} Interdependenzen\footnote{S. \pageref{interdep}}, die unter \ref{systemstruktur} genannte Systemstruktur\footnote{S. \pageref{systemstruktur}}, die unter \ref{offengeschlossen} beschriebene Offenheit und Geschlossenheit\footnote{S. \pageref{offengeschlossen}}, \ref{systemdynamik} Systemdynamik\footnote{S. \pageref{systemdynamik}}, \ref{zweckundziel} Zweck- und Zielorientierung\footnote{S. \pageref{zweckundziel}} und die unter \ref{selbsterhalt} beschriebene Selbsterhaltung\footnote{S. \pageref{selbsterhalt}} eine Rolle. Alle diese genannten Begriffe sind ihrem Wesen nach abstrakte Begriffe bzw. bedürfen erst der Abstraktion, um in einem System bestimmt oder erkannt zu werden. Die Ontologie ist daher als idealistisch bzw. konstruktivistisch einzustufen.\\
Hinsichtlich der Epistemologie liegen die Dinge bei der Systemtheorie etwas anders. Die Systemtheorie kann sowohl beschreibend und erklärend, sowie entscheidungsorientiert\footcite[S. 45]{Ulrich2016}, als auch zu Prognosen und gestaltend eingesetzt werden.\footcite[S. 47]{Ulrich2016} Da die Anwendung der Systemtheorie auch immer die oben genannte Abstraktion als Tätigkeit mit einschließt, könnte man diesen Teil als rational geprägt auffassen. Allerdings kann dann die Auseinandersetzung mit dem jeweiligen Forschungsziel auch eine eher empiristische Herangehensweise erfordern, wenn z.B. ein bestehendes Untersuchungsobjekt mit Mitteln der Systemtheorie untersucht wird. Der interdisziplinäre Ansatz der Systemtheorie macht damit die epistemologische Einordnung erst am konkreten Anwendungsfall möglich. Hans Ulrich schreibt dazu selbst: "'[...]abschließend sei bemerkt, daß m.E. der systemtheoretische Ansatz mit verschiendenen wissenschaftstheoretischen Konzeptionen vereinbar und nicht an eine bestimmte wissenschaftstheoretische Grundauffassung gebunden ist, [...]"'\footcite[S. 48]{Ulrich2016}.
\section{Kritische Einwände}
\subsection{Formalismus und empirischer Gehalt}
Eine an der Systemtheorie geäußerte Kritik, ist die des überbordenden Formalismus und damit einhergehend einer empirischen Inhaltslosigkeit.\footcite[S. 155]{Diesner2015} Danach ist die Interdisziplinarität des systemtheoretischen Ansatzes dadurch erkauft, dass man  anstelle konkret empirisch erfahrbarer Objekte formalisierte Systembegriffe setzt.\footcite[S. 23]{Oelsnitz1994} Richtig ist, dass diese Begriffe natürlich vorerst relativ bedeutungsfrei sind. Diese Bedeutung kommt erst wieder hinzu, wenn an die im System verwendeten Elemente weitere Modelle angeschlossen werden. Hans Ulrich selbst geht ebenfalls diesen Weg, indem er nach der Definition der Systemtheorie, die Verallgemeinerung wieder reduziert und zunächst auf den Untersuchungsgegenstand Betriebswirtschaftslehre\footcite[S. 135]{Ulrich1968} und weiter auf Begriffe wie z.B. Unternehmensorganisation\footcite[212]{Ulrich1968} oder die Entwicklung von Marktleistungen\footcite[299]{Ulrich2016} konkretisiert. Diese Eigenschaft wurde bereits in \ref{einordnung} Wissenschaftstheoretische Einordnung\footnote{S. \pageref{einordnung}} besprochen. "'In diesem Sinne konstatieren Lenk et al.: 'Sie (die Systemtheorie) 'ist selbst ja keine wissenschaftliche Aussage' im Sinne einer erfahrungswissenschaftlichen Theorie, 'sondern ein Mittel zum Zweck der Gewinnung und Ordnung wissenschaftlicher Erkenntnis, d.h. ein instrumenteller Ansatz mit operativen Modellen ... ' (51) Die Systemtheorie kann von daher lediglich als Vorstufe für die Ableitung speziellerer Theorien dienen'"\footcite[S. 24]{Oelsnitz1994}.
\subsection{Abstrakte Begriffswelt}
Die bisherigen Ausführungen zeigen, dass sich die Systemtheorie einer sehr abstrakten Begriffswelt bedient, die verstanden und richtig angewendet werden muss. Auch der oben genannte Kritikpunkt macht die Notwendigkeit klar, bei der Anwendung der Systemtheorie an die systemtheorischen Begriffe weitere Modelle zu knüpfen, die selbst auch wissenschaftlichen Ansprüchen genügen müssen. Eine Vorgehensweise die ausschließlich die Systemtheorie zur Methode hätte, wäre schwer umzusetzen.
\subsection{Uneinheitlichkeit}
Im Verlauf dieser Arbeit ist bereits an etlichen Stellen eingeflossen, dass der Begriff der Systemtheorie leider nicht einheitlich definiert ist. Allzu unterschiedlich sind die Ansätze und zu weit verzweigt die möglichen Entstehungslinien der einzelenen Theorievarianten der Systemtheorie. Zwar finden sich in der allgemeinen Systemtheorie viele wichtige Vorarbeiten zu den speziellen Varianten wieder. Aber, wie in den beiden Punkten oben beschrieben, trägt jede Spezialisierung wieder empirischen Gehalt ein und damit scheint die Universalität und Interdisziplinarität der Systemtheorie ihr hier selbst zum Nachteil zu werden, da die Autoren oftmals darauf verzichten, die Konkretisierung durch einen speziellen Namen deutlich zu machen.\footcite[S. 12]{Luhmann1999} Die Konsequenz daraus sollte sein, bei der Anwendung der Systemtheorie klar zu stellen auf welche Variante man sich in der jeweiligen Arbeit bezieht, um dem Leser die Einordnung des Begriffs zu erleichtern. Ggf. sollten grundlegende Begriffe nochmals beschrieben werden, insbesondere wenn sie sich in einer Variante speziell ausprägen. Es kann nicht immer davon ausgegangen werden, dass jedem Leser alle Feinheiten bekannt sind.


\section{Fazit und Zusammenfassung}

\subsection{Fazit und kritische Würdigung}
Die Systemtheorie im Allgemeinen und hier im Speziellen nach Hans Ulrich stellt einen soliden Rahmen für die Ableitung spezieller Theorien bereit.\footcite[S. 24]{Oelsnitz1994} Um dieser Aufgabe gerecht zu werden, braucht es aber gute Kenntnisse wissenschaftstheortischer Grundlagen, um die zielorientierte Integration konkretisierender Modelle in die formellen und idellen Begriffe der Systemtheorie bewerkstelligen zu können, um die ausgedehnte Anwendbarkeit der Systemtheorie in epistemologischer Hinsicht gewährleisten zu können.\\
Insbesondere für das interdisziplinäre und integrierende Arbeiten bietet die Systemtheorie durch ihre Offenheit für eine konkretere Ausprägung eine gute Basis, da diese es erlaubt auch unterschiedliche Konzepte aus unterschiedlichen Fachbereichen in ein eigenes System zu integrieren.\\
Diese Universalität ist aber Chance und Risiko zugleich. Die Gefahr besteht sich zum einen im Formalismus zu verlieren und zum anderen bei fehlender Integration in ein Gesamtsystem ein unüberschaubares Feld von unzusammenhängenden Elementen zu erzeugen.\\
Im Kontext dieser Arbeit hat sich auch schon gezeigt, welche Schwierigkeiten sich auftun die unterschiedlichen Begriffe aus den unterschiedlichen Varianten der Systemtheorie sinnvoll zusammenzuhalten. Sich immer nur auf genau eine Variante zu beschränken, kann nicht immer gelingen und es braucht Fingerspitzengefühl bei einer sinnvollen Gesamtkonzeption zu bleiben und sich auf die wesentlichen Punkte einzugrenzen, da das Feld der Systemtheorie sehr umfangreich ist und auch deutlich mehr Raum einnehmen könnte als in dem gegebenen Rahmen. Dies ist hoffentlich in dieser Arbeit gelungen.

\subsection{Ausblick}
Es sind noch viele Aspekte der Systemtheorie offen geblieben. Zum einen gibt es mitterweile viele Anwendungen dieser Systemtheorie, die eigene Arbeiten wert wären oder in andren Arbeiten aufgehen könnten, wie z.B. das Systemische Management\footcite{Malik2009} oder das St. Gallener Management-Modell\footcite{RueeggStuerm2020}.\\
Einige Aspekte haben aktuell Hochkonjunktur. So geben die Argumentationspfade zur Selbsterhaltung in \ref{selbsterhalt}\footnote{S. \pageref{selbsterhalt}} gute Hinweise zur Organisationsgestaltung in einer immer disruptiveren Welt, die starke Veränderung der Führungssysteme fordert, um Anforderungen einer VUCA Welt gerecht zu werden.\\
Für die Digitalisierung bietet die Interdisziplinarität der Systemtheorie gute Ansätze. Über die Analyse bestehender Produkte, Prozesse oder Dienstleistungen können mit Hilfe der Systemtheorie neue Systeme über funktionale Äquivalenzen abgeleitet werden, die neue strategische Potenziale aufzeigen können.\footcite[S. 65]{Hartmann2018}


% References
\newpage


% \nocite{*}
\printbibliography
% \bibliographystyle{plain}
% \bibliography{literatur}
% End of Document
\end{document}